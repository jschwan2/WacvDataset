\section{Conclusion}
We present a novel dataset intended to help researchers progress and refine their approach and ability to produce more robust RGB-D object trackers for motion parameter estimation, specifically the velocity of the object being tracked with a meaningful metric. We identify several drawbacks, and limitations with the existing methods and datasets currently available for working on these problems in addition to explaining the differences between our dataset and ground truths. We explain how we derive and ensure accuracy in our bounding boxes / object segmentation and velocity of the object by using state of the art RGB-D object segmentation as our baseline and using commonly known physics equations and set-ups for several scenes. This dataset is also the first effort in our knowledge that directly approaches these problems and we intend to keep improving and expanding upon it.

In the future we plan to extend the dataset and labeling to contain more diverse settings, classes of objects as well as their motion paths and component / limb key points. We also hope to include additional motion parameters within the labeling such as angular velocity, and rotation in degrees. These additional motion parameters have the possibility of dramatically increasing the number of possible applications that a functioning motion parameter estimator could be applied to.
